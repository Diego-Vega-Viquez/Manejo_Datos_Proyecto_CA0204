\documentclass[11pt]{letter}
\usepackage[spanish]{babel}
\usepackage[utf8]{inputenc}
\usepackage[hidelinks]{hyperref}
\usepackage{geometry}

\geometry{
    left=2.0cm, % El margen izquierdo mide 2.0cm
    right=2.0cm, % El margen derecho mide 2.0cm
    top=1.5cm, % El margen superior mide 2.0cm
    bottom=2.5cm % El margen inferior mide 2.0cm
}

\signature{José Carlos Quintero Cedeño}
\address{José Carlos Quintero Cedeño\\ jose.quinterocedeno@ucr.ac.cr}
\date{\today}

\begin{document}

\begin{letter}{Comité Editorial\\ \textit{Revista Serenguetti}\\ Escuela de Estadística\\ Universidad de Costa Rica}

\opening{Estimado comité editorial:}

Por este medio, les solicito respetuosamente considerar el siguiente manuscrito en la \textit{Revista Serenguetti} de la Escuela de Estadística (Universidad de Costa Rica). Además, en CC se está depositando una copia a Adlyceum y el DOI del mismo es \href{https://doi.org/10.5281/zenodo.14194804}{https://doi.org/10.5281/zenodo.14194804}.

El presente estudio se titula \textit{Relación entre la educación, la condición socioeconómica y la distribución del ingreso en Costa Rica, 2023}. Este estudio se sitúa en un contexto en el que nuestro país enfrenta desafíos respecto a la calidad y acceso a la educación, especialmente para las poblaciones más vulnerables, debido a un retroceso en la inversión educativa en los últimos años. Esta situación ha reducido el financiamiento per cápita, con implicaciones significativas para la equidad social y la movilidad económica a través de la educación. En particular, responde a la pregunta de cómo el nivel educativo se relaciona con la situación socioeconómica de las personas en Costa Rica y cómo esta relación puede influir en la distribución de ingresos y la desigualdad social.

Se realizaron métodos de análisis cuantitativo y descriptivo sobre datos de la ENAHO 2023 para explorar la relación entre educación, ingresos y condición socioeconómica. Se observó una correlación significativa donde niveles educativos más altos se asocian a mejores ingresos y menor probabilidad de pobreza. Factores como el nivel de instrucción y tipo de centro educativo influyen en la situación económica. Esto destaca a la educación como motor de movilidad social y reducción de desigualdades. No obstante, persisten barreras al acceso, subrayando la necesidad de políticas que mejoren la equidad educativa.

Consideramos que este trabajo es valioso para los lectores de la \textit{Revista Serenguetti}, porque ofrece un análisis actualizado sobre cómo la educación influye en la movilidad social y la desigualdad económica en Costa Rica. Además, evidencia cómo las barreras en el acceso educativo afectan la distribución de oportunidades y el bienestar socioeconómico. Esto es importante para el campo de la política pública y la educación, ya que aporta datos clave que pueden guiar decisiones para mejorar la equidad educativa y promover un desarrollo económico más inclusivo.

Para efectos administrativos, el estudiante José Carlos Quintero Cedeño (\href{mailto:jose.quinterocedeno@ucr.ac.cr}{jose.quinterocedeno@ucr.ac.cr}) es el autor para toda correspondencia.

Confirmamos que este manuscrito no ha sido publicado en otro lugar y no está siendo considerado por otra revista. Todos los autores han aprobado el manuscrito y están de acuerdo con su presentación a la \textit{Revista Serenguetti}.
\vfill
{Le saluda cordialmente,}
\vfill
José Carlos Quintero Cedeño

\end{letter}

\end{document}
